% tla2tex-example-tla.tex

% You can adjust the borders to make all your code visible.
% border={left top right bottom} or 
% border={left&right top&bottom} or
% border=length for all sides 

% Note that there are no spaces before&after ``=''.
% border={5pt 1pt}: it is not ``border = {5pt 1pt}''. 

% Note: I have set border={5pt 0pt 5pt 1pt}.
% Be careful if you want to modify it: some long lines may be cut off.
% So, use short lines in your TLA+ code.
\documentclass[preview, border={5pt 0pt 5pt 1pt}]{standalone}

% for tlaplus
\usepackage[utf8]{inputenc}
\usepackage[T1]{fontenc}
\usepackage{tlatex}
\usepackage{color}
\definecolor{boxshade}{gray}{.85}
\setboolean{shading}{true}

\begin{document}
\begin{tla}
-------------------------------- MODULE GCD --------------------------------
EXTENDS Integers
----------------------------------------------------------------------------
Divides(p, n) == \E q \in Int : n = q * p
DivisorsOf(n) == {p \in Int : Divides(p, n)}

SetMax(S) == CHOOSE i \in S : \A j \in S : i >= j

GCD(m, n) == SetMax(DivisorsOf(m) \cap DivisorsOf(n))  \* gcd of $m$ and $n$
SetGCD(T) == SetMax({d \in Int : \A t \in T : Divides(d, t)})
=============================================================================
\end{tla}
\begin{tlatex}
\@x{}\moduleLeftDash\@xx{ {\MODULE} GCD}\moduleRightDash\@xx{}%
\@x{ {\EXTENDS} Integers}%
\@x{}\midbar\@xx{}%
 \@x{ Divides ( p ,\, n ) \.{\defeq} \E\, q \.{\in} Int \.{:} n \.{=} q \.{*}
 p}%
 \@x{ DivisorsOf ( n ) \.{\defeq} \{ p \.{\in} Int \.{:} Divides ( p ,\, n )
 \}}%
\@pvspace{8.0pt}%
 \@x{ SetMax ( S ) \.{\defeq} {\CHOOSE} i \.{\in} S \.{:} \A\, j \.{\in} S
 \.{:} i \.{\geq} j}%
\@pvspace{8.0pt}%
 \@x{ GCD ( m ,\, n ) \.{\defeq} SetMax ( DivisorsOf ( m ) \.{\cap} DivisorsOf
 ( n ) )\@s{4.1}}%
\@y{%
  gcd of $m$ and $n$
}%
\@xx{}%
 \@x{ SetGCD ( T ) \.{\defeq} SetMax ( \{ d \.{\in} Int \.{:} \A\, t \.{\in} T
 \.{:} Divides ( d ,\, t ) \} )}%
\@x{}\bottombar\@xx{}%
\end{tlatex}
\end{document}